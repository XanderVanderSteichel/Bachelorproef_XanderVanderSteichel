%%=============================================================================
%% Voorwoord
%%=============================================================================

\chapter*{\IfLanguageName{dutch}{Woord vooraf}{Preface}}%
\label{ch:voorwoord}

%% TODO:
%% Het voorwoord is het enige deel van de bachelorproef waar je vanuit je
%% eigen standpunt (``ik-vorm'') mag schrijven. Je kan hier bv. motiveren
%% waarom jij het onderwerp wil bespreken.
%% Vergeet ook niet te bedanken wie je geholpen/gesteund/... heeft

Dit project heeft mij de kans gegeven om mijn interesse in zowel e-commerce als 3D-printtechnologie te combineren. Vanwege de steeds grotere vraag naar geautomatiseerde oplossingen ook voor mezelf, was ik benieuwd naar de mogelijkheden om een bestelling om te zetten in een 3D-geprint product. Deze bachelorproef is om het potentieel van CI/CD-tools zoals Jenkins, GitHub Actions en GitLab CI te verkennen, niet alleen voor softwareontwikkeling, maar ook voor de fysieke wereld van 3D-printen.
\\\\
Tijdens het ontwikkelen van de verschillende proof-of-concepts kwam ik vaak voor technische uitdagingen te staan, maar elke uitdaging bood me de mogelijkheid om mijn vaardigheden verder te ontwikkelen. Het project heeft me niet alleen nieuwe inzichten opgeleverd in het gebruik van APIs en automatiseringssystemen, maar ook in hoe technologie praktische problemen in de industrie kan oplossen.
\\\\
Ik wil mijn begeleider bedanken voor zijn waardevolle feedback en steun gedurende dit proces. Daarnaast wil ik mijn familie en vrienden bedanken voor hun geduld en motivatie, vooral in de intensievere momenten van het project. Zonder hun hulp zou deze bachelorproef niet het resultaat zijn geworden dat het nu is.
\\\\
Tot slot ben ik dankbaar voor de technische middelen en tools die beschikbaar waren om dit project mogelijk te maken, en ik kijk ernaar uit om de kennis in de praktijk toe te passen.

