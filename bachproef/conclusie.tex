%%=============================================================================
%% Conclusie
%%=============================================================================

\chapter{Conclusie}%
\label{ch:conclusie}

% TODO: Trek een duidelijke conclusie, in de vorm van een antwoord op de
% onderzoeksvra(a)g(en). Wat was jouw bijdrage aan het onderzoeksdomein en
% hoe biedt dit meerwaarde aan het vakgebied/doelgroep? 
% Reflecteer kritisch over het resultaat. In Engelse teksten wordt deze sectie
% ``Discussion'' genoemd. Had je deze uitkomst verwacht? Zijn er zaken die nog
% niet duidelijk zijn?
% Heeft het onderzoek geleid tot nieuwe vragen die uitnodigen tot verder 
%onderzoek?

Dit bachelorproject onderzocht hoe een e-commerce platform zoals Shopify automatisch fysieke producten kan laten printen via een 3D-printer, zonder handmatige tussenkomst. Concreet werd onderzocht hoe een CI/CD-pipeline, zoals Jenkins, GitHub Actions of GitLab CI, gebruikt kan worden als brug tussen een webshop en een 3D-printer die aangestuurd wordt via Moonraker.

Om dit te realiseren werden drie proof-of-concepts (PoC's) uitgewerkt:

\begin{itemize}
    \item \textbf{PoC 1} toonde aan dat Jenkins succesvol 3D-printtaken kan starten op basis van een Shopify-webhook. De pipeline uploadt G-code bestanden, start de printerqueue en wacht actief tot de print voltooid is.
    
    \item \textbf{PoC 2} bewees dat GitHub Actions ook geschikt is voor deze automatisatie, mits een externe tussenlaag (zoals een Express.js server) de webhooks opvangt en doorstuurt naar GitHub.
    
    \item \textbf{PoC 3} exploreerde de mogelijkheden met GitLab CI, met een focus op eenvoud en betere integratie-opties. GitLab toont gelijkaardige flexibiliteit.
\end{itemize}

De meest mature oplossing op technisch vlak was \textbf{PoC 1}, die het volledige proces van bestelling tot geprinte output afhandelt. Dankzij Jenkins' configuratie om builds niet parallel uit te voeren (\texttt{Do not allow concurrent builds}), en wordt elke bestelling in volgorde verwerkt. De pipeline wacht per product actief tot de print afgerond is, waardoor betrouwbaarheid en traceerbaarheid gegarandeerd zijn.

\section*{Bijdrage aan het domein}

Het project toont aan dat CI/CD-tools niet enkel nuttig zijn voor softwaredeployments, maar ook als coördinatieplatform kunnen dienen voor fysieke processen zoals 3D-printing. Deze benadering is relatief ongezien, en biedt een laagdrempelige automatisatie-oplossing aan voor kleine ondernemingen die geen toegang hebben tot complexe productielijnen of gespecialiseerde software.

\section*{Kritische reflectie}

De pipelines functioneren stabiel en robuust, maar vereisen wel enige technische setup, vooral qua netwerkinstellingen (zoals ngrok of port forwarding). In een productiecontext zou een meer beveiligde oplossing vereist zijn zoals het lokaal draaien op een externe Rasberry Pi.
\\\\
Ook blijft foutafhandeling aan de printerkant een uitdaging. Hoewel de pipeline printerfouten detecteert via de Moonraker API, kunnen problemen zoals mechanische falen of slecht gelevelde bedden enkel deels worden opgevangen. Moest er een camera aangesloten zijn zou er al geprobeerd kunnen worden om er AI detectie op te  zetten. Dit wordt bijvoorbeeld bij de Bambulab toegepast.

\section*{Vervolgonderzoek}

Er zijn meerdere opportuniteiten voor vervolgonderzoek:

\begin{itemize}
    \item Integratie met meerdere printers tegelijk, en load balancing.
    \item Integratie met order- en voorraadbeheer (bv. automatisch printen bij lage stock).
    \item Meer verfijnde foutafhandeling en recovery-opties (bv. notificaties via mail of Discord).
    \item Een visuele interface om printstatussen op te volgen als je meerdere printers hebt.
\end{itemize}

\bigskip

Tot slot toont dit project aan dat zelfs met bestaande tools zoals Jenkins en enkele scripts, een betrouwbare, flexibele en schaalbare printautomatisatie mogelijk is – een brug tussen digitale bestellingen en fysieke productie.


