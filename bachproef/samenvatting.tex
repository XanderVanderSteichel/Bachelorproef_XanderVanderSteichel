%%=============================================================================
%% Samenvatting
%%=============================================================================

% TODO: De "abstract" of samenvatting is een kernachtige (~ 1 blz. voor een
% thesis) synthese van het document.
%
% Een goede abstract biedt een kernachtig antwoord op volgende vragen:
%
% 1. Waarover gaat de bachelorproef?
% 2. Waarom heb je er over geschreven?
% 3. Hoe heb je het onderzoek uitgevoerd?
% 4. Wat waren de resultaten? Wat blijkt uit je onderzoek?
% 5. Wat betekenen je resultaten? Wat is de relevantie voor het werkveld?
%
% Daarom bestaat een abstract uit volgende componenten:
%
% - inleiding + kaderen thema
% - probleemstelling
% - (centrale) onderzoeksvraag
% - onderzoeksdoelstelling
% - methodologie
% - resultaten (beperk tot de belangrijkste, relevant voor de onderzoeksvraag)
% - conclusies, aanbevelingen, beperkingen
%
% LET OP! Een samenvatting is GEEN voorwoord!

%%---------- Nederlandse samenvatting -----------------------------------------
%
% TODO: Als je je bachelorproef in het Engels schrijft, moet je eerst een
% Nederlandse samenvatting invoegen. Haal daarvoor onderstaande code uit
% commentaar.
% Wie zijn bachelorproef in het Nederlands schrijft, kan dit negeren, de inhoud
% wordt niet in het document ingevoegd.

\IfLanguageName{english}{%
\selectlanguage{dutch}
\chapter*{Samenvatting}
\lipsum[1-4]
\selectlanguage{english}
}{}

%%---------- Samenvatting -----------------------------------------------------
% De samenvatting in de hoofdtaal van het document

\chapter*{\IfLanguageName{dutch}{Samenvatting}{Abstract}}

 Dit bachelorproefvoorstel richt zich op de automatisering van het orderproces van website naar 3D-geprinte producten. Het onderzoekt hoe een Python-script aan de hand van verschillende pipelines met API-aanroepen, orders kan verwerken die vanuit een Shopify-webshop automatisch doorgestuurd worden naar de Bambu Lab X1C 3D-printer. Het probleem dat nu vaak voorkomt, is dat dit manueel moet gebeuren, wat inefficiënt en foutgevoelig kan zijn. Hierdoor komt de onderzoeksvraag: "Hoe kan een geautomatiseerde pipeline worden ontwikkeld die orders vanuit een website automatisch omzet naar een opdracht voor een 3D-printer?" Het onderzoek is vooral gericht op het vergelijken van de verschillende soorten pipelines en het bieden van hulp aan kleine bedrijven om op een goedkopere manier automatisatie aan te bieden. Het resultaat zal een werkend Python-script zijn dat Shopify API en Bambu Studio API integreert. De methodologie omvat een literatuurstudie naar bestaande pipelines en API-integraties, de ontwikkeling van een proof-of-concept (PoC), en uitgebreide tests om prestaties en betrouwbaarheid te verkrijgen. Dit project zal zich specifiek richten op uitdagingen zoals API-limieten, foutafhandeling, en schaalbaarheid. Wat verwacht wordt is dat het eindproduct de productietijd versnelt, de foutmarge verlaagt, en kleine bedrijven helpt hun focus te verleggen naar andere activiteiten.
